\documentclass[12pt]{article}

\usepackage[backend=bibtex]{biblatex}
\bibliography{main}

\begin{document}

Authors:

Customer:

Instructor:

\section{{}{Introduction (Due Nov. 16)}}\label{introduction-due-nov.-16}

Placeholder~\cite{alaettinoglurfc}.

Placeholder~\cite{blunk2005rfc}.

\begin{itemize}
\item
  Provide an overview of the entire SRS subsections
\item
  Indicate the topics that will be covered in this document.
\end{itemize}

Start of your text.

\subsection{Purpose}\label{purpose}

\begin{itemize}
\item
  What's the purpose of the SRS document?
\item
  Specify the intended audience.
\end{itemize}

Start of your text.

\subsection{Scope}\label{scope}

\begin{itemize}
\item
  Identify SW product(s) to be produced by name
\item
  Describe the application of SW being specified, including benefits,
  objectives, goals. What is the application domain? (e.g., embedded
  system for automotive systems, graphical modeling utility) This is the
  domain description of the application.
\item
  Explain what SW product will, and if necessary, will not do. This is
  the requirement of the application.
\item
  Be consistent with similar statements in higher-level specifications
  (e.g., the original project specification from customer)
\end{itemize}

\subsection*{}\label{section}
\addcontentsline{toc}{section}{}

Start of your text.

\subsection{Definitions, acronyms, and
abbreviations}\label{definitions-acronyms-and-abbreviations}

\begin{itemize}
\item
  Define all terns, acronyms, and abbreviations need to understand the
  SRS. If this section is extensive, then move to an appendix. It is
  also possible to provide a link to other resources for extensive
  terminology explanation.
\end{itemize}

Start of your text.

\subsection{Organization}\label{organization}

\begin{itemize}
\item
  Describe what the rest of the SRS contains
\item
  Give the organizational structure of the SRS.
\end{itemize}

Start of your text.

\section{{Overall Description \emph{(Due Nov.
16)}}{Overall Description (Due Nov. 16)}}\label{overall-description-due-nov.-16}

\begin{itemize}
\item
  Give a brief introduction of what information will be covered in this
  section.
\end{itemize}

Start of your text.

\section{Product Perspective}\label{product-perspective}

\begin{itemize}
\item
  Describe the context for the product
\item
  Is it one element that is part of a bigger system? If so, then give a
  pictoral representation or diagram (e.g., data flow diagram -- DFD,
  block diagram) that describes how your product fits.
\item
  Interface Constraints:

  \begin{itemize}
  \item
    System interfaces
  \item
    User interfaces
  \item
    HW interfaces
  \item
    SW interfaces
  \item
    Communication interfaces
  \end{itemize}
\item
  Other types of constraints:

  \begin{itemize}
  \item
    Memory
  \item
    Operations
  \item
    Site adaptation operations (customization that is done on-site).
  \end{itemize}
\end{itemize}

Start of your text.

\section{Product Functions}\label{product-functions}

\begin{itemize}
\item
  Summarize the major functions that software will perform (portions may
  come directly from the customer specification -- cite as appropriate).
\item
  These function descriptions should be easily understandable by the
  customer or to any general reader.
\item
  Diagrams: (for all diagrams, introduce the notation first)

  \begin{itemize}
  \item
    Give and describe a high-level goal diagram for system.
  \end{itemize}
\end{itemize}

Start of your text.

\section{User Characteristics}\label{user-characteristics}

\begin{itemize}
\item
  Expectations about the user (e.g., background, skill level, general
  expertise)
\end{itemize}

Start of your text.

\section{Constraints}\label{constraints}

\begin{itemize}
\item
  See list of possible constraints from IEEE SRS document.
\item
  Give English descriptions of safety-critical properties
\item
  Give English descriptions of other properties that if violated, the
  system will not perform properly.
\item
  For at least two of the properties, create a SCR specification that
  specifies how the relevant system elements should behave in order to
  satisfy the property. (Recall that SCR is most effective when you have
  complex logic for handling events; select the properties
  appropriately.)

  \begin{itemize}
  \item
    Create event, condition, and mode transition tables (explain)
  \end{itemize}
\end{itemize}

Start of your text.

\section{Assumptions and
Dependencies}\label{assumptions-and-dependencies}

\begin{itemize}
\item
  Assumptions made about the HW, SW, environment, user interactions.
\end{itemize}

Start of your text.

\section{Approportioning of
Requirements}\label{approportioning-of-requirements}

\begin{itemize}
\item
  Based on negotiations with customers, requirements that are determined
  to be beyond the scope of the current project and may be addressed in
  future versions/releases.
\end{itemize}

Start of your text.

\section{{Specific Requirements \emph{(Due Nov.
16)}}{Specific Requirements (Due Nov. 16)}}\label{specific-requirements-due-nov.-16}

\begin{itemize}
\item
  Give an enumerated list of requirements.
\item
  As appropriate, use a hierarchical numbering scheme.
\end{itemize}

\begin{enumerate}
\def\labelenumi{\arabic{enumi}.}
\item
  Sample requirement at the top level

  \begin{enumerate}
  \def\labelenumii{\arabic{enumii}.}
  \item
    Level 2 requirement example
  \item
    Another Level 2 requirement
  \end{enumerate}
\item
  Select the ``Requirement'' Style.
\end{enumerate}

\section{{Modeling Requirements
\emph{(1\textsuperscript{st} draft: Nov. 28; Final draft: Dec.
7)}}{Modeling Requirements (1st draft: Nov. 28; Final draft: Dec. 7)}}\label{modeling-requirements-1st-draft-nov.-28-final-draft-dec.-7}

\begin{itemize}
\item
  This is the specification portion of the requirements document.
  (Specifying the bridge between the application domain and the machine
  domain.)

  \begin{itemize}
  \item
    Give and describe use case diagrams
  \end{itemize}
\item
  Check: Every goal at the bottom of the goal model should be addressed
\item
  Each goal may be satisfied by 1 or more use cases
\item
  Each use case should refer to 1 or more requirements (in Section 4)

  \begin{itemize}
  \item
    Give and describe a high-level class diagram that depicts the key
    elements of the system
  \item
    Representative Scenarios of System:

    \begin{itemize}
    \item
      Give English descriptions of representative scenarios for each use
      cases.
    \end{itemize}
  \end{itemize}
\item
  Check: use instances of the class names from class diagram; refer to
  the terms used in use case diagram

  \begin{itemize}
  \item
    For each scenario, give a corresponding sequence diagram
  \end{itemize}
\item
  Check: Objects should be instances of classes in class diagram

  \begin{itemize}
  \item
    Create and explain a state diagram for all key classes that
    participate in the scenarios (from above).
  \end{itemize}
\item
  Check: that all scenarios can be validated against the state diagrams.
\item
  Check that the events, actions are modeled in the class diagram.
\item
  Check that all variables referenced in the diagrams are declared as
  attributes in the class diagram.
\end{itemize}

Start of your text.

\section{{Prototype \emph{(Due Nov. 27, 5:00
p.m.)}}{Prototype (Due Nov. 27, 5:00 p.m.)}}\label{prototype-due-nov.-27-500-p.m.}

\begin{itemize}
\item
  Describe what your prototype will show in terms of system
  functionality.
\end{itemize}

\section{How to Run Prototype}\label{how-to-run-prototype}

\begin{itemize}
\item
  Describe what is needed to run your prototype
\item
  What system configuration? (Should be accessible through web.) Are
  there plugins? Are there any OS or networking constraints. Give the
  URL for the prototype (if running from your website). Minimally, there
  should be a prototype link on your webpage that is executable.
\end{itemize}

\section{Sample Scenarios}\label{sample-scenarios}

\begin{itemize}
\item
  Give a sample scenario of using your system. Use real data and problem
  scenarios. Include screen captures illustrating what your prototype
  produces. As always, be sure to describe all figures.
\end{itemize}

\section{References}\label{references}

\begin{itemize}
\item
  Provide list of all documents referenced in the SRS
\item
  Identify each document by title, report number, date, and publishing
  organization.
\item
  Specify the sources from which the references can be obtained.
\item
  Include an entry for your project website.
\end{itemize}

Start of your text.

\begin{enumerate}
\def\labelenumi{\arabic{enumi}.}
\item
  D. Thakore and S. Biswas, ``Routing with Persistent Link Modeling in
  Intermittently Connected Wireless Networks,'' Proceedings of IEEE
  Military Communication, Atlantic City, October 2005.
\end{enumerate}

\section{Point of Contact}\label{point-of-contact}

For further information regarding this document and project, please
contact \textbf{Prof. Betty H.C. Cheng} at Michigan State University
(chengb at cse.msu.edu). All materials in this document have been
sanitized for proprietary data. The students and the instructor
gratefully acknowledge the participation of our industrial
collaborators.

\printbibliography

\end{document}
