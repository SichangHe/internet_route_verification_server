\documentclass[12pt]{article}

\usepackage[backend=bibtex]{biblatex}
\bibliography{main}

\title{Software Requirement Specification for Route Verification Server}
\author
    {Authors: Sichang He (Steven), Shouju Wang\\
    Customer: Dr. Italo Cunha\\
    Instructor: Dr. Mustafa Misir
}

\begin{document}
\maketitle

\section{Introduction}
% Identify SW product(s) to be produced by name
We shall produce a REST API server \verb|route_verification_server| and
a Python client library \verb|route_verification_client| for
convenient requests to the server.

% Describe the application of SW being specified, including benefits,
% objectives, goals. What is the application domain? (e.g., embedded
% system for automotive systems, graphical modeling utility) This is the
% domain description of the application.
This software is used for networks research, operation, and maintenance.
Users shall be able to query for Internet routing policies and
policy verification reports based on various conditions they provide.

% Explain what SW product will, and if necessary, will not do. This is
% the requirement of the application.
\verb|route_verification_server| will store routing policies and
verification reports in its internal database.
It will also respond to query requests by querying the database.
However,
\verb|route_verification_server| will not collect routing policies or
generate verification reports.
\verb|route_verification_client| will request the server for information
using its REST API,
but it will not directly query the server's internal database.

% Define all terns, acronyms, and abbreviations need to understand the
% SRS. If this section is extensive, then move to an appendix. It is
% also possible to provide a link to other resources for extensive
% terminology explanation.

Public routing policies are retrieved from
the Routing Policy Specification Language
(RPSL)~\cite{alaettinoglurfc}~\cite{blunk2005rfc} in
the Internet Route Registry (IRR)~\cite{irr2023}
Specified with the RPSL,
each Autonomous System (AS) on the Internet has its policies to
accept or reject routes when importing or exporting them.

The University of Oregon Route Views Archive Project~\cite{route2023}
collects observed routes on the Internet.
Each observed route contains the IP prefix being propagated,
the AS path the route traverses through,
and other information.
The AS path is a list of integers representing ASes,
with the AS on the right exporting the IP prefix to its neighbor on
its left,
which imports it.
\verb|route_verification_server| will store these observed routes for
querying.

The CAIDA Data Server provides an AS Relationship Database.~\cite{index2023}
The database specifies peer-to-peer (P2P) and provider-to-customer (P2C)
relationships between AS pairs.
It also lists out the tier-1 provider (clique) ASes.
These information about the AS hierarchy and business relationships helps
identify and explain special cases in observed routes.
\verb|route_verification_server| will store a copy of the
AS Relationship Database used.

The Internet Route Verification project~\cite{internet2023he}
compares observed routes with public routing policies and
generates verification reports.
To enable the comparison,
it also parses the policies into intermediate representation (IR).
For each route,
each import and export between two ASes is verified using
their recorded policies;
one report is generated for each import/export.
When mismatches are encountered,
the Internet Route Verification project uses
CAIDA's AS Relationship Database to help determine special cases.
\verb|route_verification_server| will store these IR and
verification reports for querying.

\section{User requirements definition}
% \subsection{Product Perspective}

% Describe the context for the product
In a changing Internet environment,
network operators and researchers need tools to monitor both
the routing policies they publish in the IRR and
the actual routes permitted to propagate.
The RPSL used to specify routing policies has highly relational semantics,
where ASes, routes, and other objects are tightly coupled.
To retrieve the information necessary to identify network regularities and anomalies,
A server to model the relationships and query routing information is needed.

The interface of \verb|route_verification_server| needs to
be simple and flexible for maintainability.
For this purpose,
we shall design a REST API endpoint for each kind of query,
and accompany it by \verb|route_verification_client|,
a Python client library for programmable client interaction.
It also needs to be efficient enough to run on average personal computers.

% \subsection{Product Functions}
% Summarize the major functions that software will perform (portions may
% come directly from the customer specification -- cite as appropriate).

% These function descriptions should be easily understandable by the
% customer or to any general reader.
\verb|route_verification_server| shall have the following functions from
the users' perspective.

\begin{itemize}
    \item Store RPSL objects from the IRR,
    including ASes, AS Sets, Route objects, Route Sets, Filter Sets,
    Peering Sets, and Maintainers.
    Each RPSL object has a name and body.
    Specific RPSL objects have extra information attached.
    \begin{itemize}
        \item An AS has an AS number, AS name, and
        multiple AS Sets it belongs to.
        \item An AS Set contains member AS numbers and AS Sets, and
        can have a list of Maintainers whose ASes belong to the AS Set.
        \item A Route object has an IP address prefix, an origin, and
        a multiple Route Sets it belongs to.
        \item A Maintainer object has a description,
        multiple other Maintainer names that maintains it,
        date last modified, and source database.
    \end{itemize}
    \item Store observed routes retrieved from the University of
    Oregon Route Views Archive Project.
    Each observed route has a pipe-separated raw form,
    an IP address prefix, an AS Path consisting of ASes,
    and a BGP Collector that has an AS number and an IP.
    \item Store AS Relationships from CAIDA,
    including provider-customer and peer-to-peer relationships.
    \item Store parsed policy IR from Internet Route Verification,
    including ASes, AS Sets, Route Sets, Peering Sets, Filter Sets, and
    AS Routes.
    \item Store verification reports from Internet Route Verification for
    observed routes.
    Each observed route correspond to multiple report.
    A report can either be an import or an export,
    has an AS it comes from and an AS it goes to,
    an overall type among OK, skip, unrecorded, special case, and error,
    and multiple report items.
    A report item has a category corresponding to
    the overall type mentioned above,
    a specific case it belongs to,
    and may have a string content or a number content associated to it.
    \item Provide item count and paging for all queries.
    \item Query RPSL objects from the IRR by name,
    including each specific type mentioned above.
    \item Query verification reports for observed routes.
    \item Query RPSL objects, routes, IR, reports,
    and report items for a given AS.
    \item Query ASes, routes, reports and report items for a given
    overal type the reports belong to.
    \item Query ASes, routes, reports and report items for a given
    specific case the report items belong to.
    \item Query reports and report items for a given Route object.
    % TODO: continue
\end{itemize}

% Diagrams: (for all diagrams, introduce the notation first)

%   Give and describe a high-level goal diagram for system.

% \subsection{User Characteristics}
% Expectations about the user (e.g., background, skill level, general
% expertise)
We assume that the users are comfortable enough with Internet routing
such that they can identify their own specific needs to
query RPSL, routes, and verification reports.
We also assume that the users are capable of Python scripting with
our provided client library to fulfill their needs.

\section{System requirements specification }
\begin{itemize}
    \item Store parsed policy IR as JSON serialization.
    % TODO: continue
\end{itemize}

\section{System Models}
% Use cases.
\paragraph{Use case 1}
A network operator wants to find out whether the server is
using their latest policies for the comparison.
They would query for the ASes they maintain directly and
check the returned RPSL object.

\paragraph{Use case 2}
A network researcher wants to check the correlation between
the number of policies an AS has and
its erroneous observed routes.
They would query all ASes, their policies,
and their verification reports by
filtering the number of policies and the type of the reports.

\paragraph{Use case 3}
A network maintainer wants to clean up the routes specified in
their Route Sets.
They would query for Route objects both in Route Sets maintained by
them but not originating from ASes they maintain.

\paragraph{Use case 4}
A network researcher wants a list of ASes using policies not implemented by
\verb|internet_route_verification|.
They would query for parsed policies and corresponding ASes by
filtering policies that contain keywords like "Community".

\printbibliography

\end{document}
